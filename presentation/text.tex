% Тип документа
\documentclass[a4paper,12pt]{extarticle}

% Шрифты, кодировки, символьные таблицы, переносы
\usepackage{cmap}
\usepackage[T2A]{fontenc}
\usepackage[utf8x]{inputenc}
\usepackage[russian]{babel}

% Это пакет -- хитрый пакет, он нужен но не нужен
\usepackage[mode=buildnew]{standalone}

\usepackage
	{
		% Дополнения Американского математического общества (AMS)
		amssymb,
		amsfonts,
		amsmath,
		amsthm,
		physics,
		% misccorr,
		% 
		% Графики и рисунки
		wrapfig,
		graphicx,
		subcaption,
		float,
		tikz,
		tikz-3dplot,
		caption,
		csvsimple,
		color,
		booktabs,
		pgfplots,
		pgfplotstable,
		geometry,
		% 
		% Таблицы, списки
		makecell,
		multirow,
		indentfirst,
		%
		% Интегралы и прочие обозначения
		ulem,
		esint,
		esdiff,
		% 
		% Колонтитулы
		fancyhdr,
	}  

\usepackage{xcolor}
\usepackage{hyperref}

 % Цвета для гиперссылок
\definecolor{linkcolor}{HTML}{000000} % цвет ссылок
\definecolor{urlcolor}{HTML}{799B03} % цвет гиперссылок
 
\hypersetup{pdfstartview=FitH,  linkcolor=linkcolor,urlcolor=urlcolor, colorlinks=true}
% Обводка текста в TikZ
\usepackage[outline]{contour}

% Увеличенный межстрочный интервал, французские пробелы
\linespread{1.3} 
\frenchspacing 

 
\usetikzlibrary
	{
		decorations.pathreplacing,
		decorations.pathmorphing,
		patterns,
		calc,
		scopes,
		arrows,
		fadings,
		through,
		shapes.misc,
		arrows.meta,
		3d,
		quotes,
		angles,
		babel
	}


\tikzset{
	force/.style=	{
		>=latex,
		draw=blue,
		fill=blue,
				 	}, 
	%				 	
	axis/.style=	{
		densely dashed,
		blue,
		line width=1pt,
		font=\small,
					},
	%
	th/.style=	{
		line width=1pt},
	%
	acceleration/.style={
		>=open triangle 60,
		draw=magenta,
		fill=magenta,
					},
	%
	inforce/.style=	{
		force,
		double equal sign distance=2pt,
					},
	%
	interface/.style={
		pattern = north east lines, 
		draw    = none, 
		pattern color=gray!60,
					},
	cross/.style=	{
		cross out, 
		draw=black, 
		minimum size=2*(#1-\pgflinewidth), 
		inner sep=0pt, outer sep=0pt,
					},
	%
	cargo/.style=	{
		rectangle, 
		fill=black!70, 
		inner sep=2.5mm,
					},
	%
	caption/.style= {
		midway,
		fill=white!20, 
		opacity=0.9
					},
	%
	}

\newenvironment{tikzpict}
    {
	    \begin{figure}[htbp]
		\centering
		\begin{tikzpicture}
    }
    { 
		\end{tikzpicture}
		% \caption{caption}
		% \label{fig:label}
		\end{figure}
    }


\newcommand{\vbLabel}[3]{\draw ($(#1,#2)+(0,5pt)$) -- ($(#1,#2)-(0,5pt)$) node[below]{#3}}
\newcommand{\vaLabel}[3]{\draw ($(#1,#2)+(0,5pt)$) node[above]{#3} -- ($(#1,#2)-(0,5pt)$) }

\newcommand{\hrLabel}[3]{\draw ($(#1,#2)+(5pt,0)$) -- ($(#1,#2)-(5pt,0)$) node[right, xshift=1em]{#3}}
\newcommand{\hlLabel}[3]{\draw ($(#1,#2)+(5pt,0)$) node[left, xshift=-1em]{#3} -- ($(#1,#2)-(5pt,0)$) }



\newcommand\zi{^{\,*}_i}
\newcommand\sumn{\sum_{i=1}^{N}}

\tikzset{
	coordsys/.style={scale=1.8,x={(1.1cm,-0cm)},y={(0.5cm,1cm)}, z={(0cm,0.8cm)}},
	coordsys/.style={scale=1.5,x={(0cm,0cm)},y={(1cm,0cm)}, z={(0cm,1cm)}}, 
	coordsys/.style={scale=1.5,x={(1cm,0cm)},y={(0cm,1cm)}, z={(0cm,0cm)}}, 
}

\usepgfplotslibrary{units}


% Draw line annotation
% Input:
%   #1 Line offset (optional)
%   #2 Line angle
%   #3 Line length
%   #5 Line label
% Example:
%   \lineann[1]{30}{2}{$L_1$}

\newcommand{\lineann}[4][0.5]{%
    \begin{scope}[rotate=#2, blue,inner sep=2pt, ]
        \draw[dashed, blue!40] (0,0) -- +(0,#1)
            node [coordinate, near end] (a) {};
        \draw[dashed, blue!40] (#3,0) -- +(0,#1)
            node [coordinate, near end] (b) {};
        \draw[|<->|] (a) -- node[fill=white, scale=0.8] {#4} (b);
    \end{scope}
}

\newcommand{\lineannn}[4][0.5]{%
    \begin{scope}[rotate=#2, blue,inner sep=2pt, ]
        \draw[dashed, blue!40] (0,0) -- +(0,#1)
            node [coordinate, near end] (a) {};
        \draw[dashed, blue!40] (#3,0) -- +(0,#1)
            node [coordinate, near end] (b) {};
        % \draw[color=white, color=blue] (a) -- node[fill=white, scale=0.8] {#4} (b);
        \draw[->|] (a)++(-0.3,0) -- (a);
        \draw[->|] (b)++(0.3,0) coordinate (xx) -- (b);
        \draw (xx) node[fill=white, scale=0.8, right] {#4};
    \end{scope}
}

% Круговая стрелка относительно центра (дуга из центра)
\tikzset{
  pics/carc/.style args={#1:#2:#3}{
    code={
      \draw[pic actions] (#1:#3) arc(#1:#2:#3);
    }
  },
  dash/.style={
  	dash pattern=on 5mm off 5mm
  }
}

% Среднее <#1>
\newcommand{\mean}[1]{\langle#1\rangle}

\pgfplotsset{
    % most recent feature set of pgfplots
    compat=newest,
}

% const прямым шрифтом
\newcommand\ct[1]{\text{\rmfamily\upshape #1}}
\newcommand*{\const}{\ct{const}}


\usepackage[europeanresistors,americaninductors]{circuitikz}

% Style to select only points from #1 to #2 (inclusive)
\pgfplotsset{select/.style 2 args={
    x filter/.code={
        \ifnum\coordindex<#1\def\pgfmathresult{}\fi
        \ifnum\coordindex>#2\def\pgfmathresult{}\fi
    }
}}


\usepackage{array}
\usepackage{pstool}


%%%%%%%%%%%%%%%%%%%%%%%%%%%%%%%%%%%%%%%%%%%%%%%%%
\makeatletter
\newif\if@gather@prefix 
\preto\place@tag@gather{% 
  \if@gather@prefix\iftagsleft@ 
    \kern-\gdisplaywidth@ 
    \rlap{\gather@prefix}% 
    \kern\gdisplaywidth@ 
  \fi\fi 
} 
\appto\place@tag@gather{% 
  \if@gather@prefix\iftagsleft@\else 
    \kern-\displaywidth 
    \rlap{\gather@prefix}% 
    \kern\displaywidth 
  \fi\fi 
  \global\@gather@prefixfalse 
} 
\preto\place@tag{% 
  \if@gather@prefix\iftagsleft@ 
    \kern-\gdisplaywidth@ 
    \rlap{\gather@prefix}% 
    \kern\displaywidth@ 
  \fi\fi 
} 
\appto\place@tag{% 
  \if@gather@prefix\iftagsleft@\else 
    \kern-\displaywidth 
    \rlap{\gather@prefix}% 
    \kern\displaywidth 
  \fi\fi 
  \global\@gather@prefixfalse 
} 
\newcommand*{\beforetext}[1]{% 
  \ifmeasuring@\else
  \gdef\gather@prefix{#1}% 
  \global\@gather@prefixtrue 
  \fi
} 
\makeatother
%%%%%%%%%%%%%%%%%%%%%%%%%%%%%%%%%%%%%%%%%%%%%%%%%

\geometry		
	{
		left			=	2cm,
		right 			=	2cm,
		top 			=	3cm,
		bottom 			=	3cm,
		bindingoffset	=	0cm
	}

%%%%%%%%%%%%%%%%%%%%%%%%%%%%%%%%%%%%%%%%%%%%%%%%%%%%%%%%%%%%%%%%%%%%%%%%%%%%%%%




%%%%%%%%%%%%%%%%%%%%%%%%%%%%%%%%%%%%%%%%%%%%%%%%%%%%%%%%%%%%%%%%%%%%%%%%%%%%%%%

\renewcommand{\contentsname}{Оглавление}

\usepackage{tocloft}
% \renewcommand{\cftpartleader}{\cftdotfill{\cftdotsep}} % for parts
% \renewcommand{\cftsectiondotsep}{\cftdotsep}% Chapters should use dots in ToC
\renewcommand{\cftsecleader}{\cftdotfill{\cftdotsep}}
%\renewcommand{\cftsecleader}{\cftdotfill{\cftdotsep}} % for sections, if you really want! (It is default in report and book class (So you may not need it).
% ---------
% \newcommand{\cftchapaftersnum}{.}%
% \usepackage{titlesec}
% \titlelabel{\thetitle.\quad}
\usepackage{secdot}
\sectiondot{subsection}
\begin{document}

\def\labauthors{}
\def\labgroup{430}
\def\labnumber{ последней}
\def\labtheme{МПЭ}

\begin{titlepage}

\begin{center}

{\small\textsc{Нижегородский государственный университет имени Н.\,И. Лобачевского}}
\vskip 1pt \hrule \vskip 3pt
{\small\textsc{Радиофизический факультет}}

\vfill

{\Large Отчет по лабораторной работе №\labnumber\vskip 12pt\bfseries \labtheme}
	
\end{center}

\vfill
	
\begin{flushright}
	{Выполнили студенты \labgroup\ группы\\ \labauthors}%\vskip 12pt Принял:\\ Менсов С.\,Н.}
\end{flushright}
	
\vfill
	
\begin{center}
	Нижний Новгород, \the\year
\end{center}

\end{titlepage}



\section*{Слайд 1}
Представление работы

\section*{Слайд 2}
Для начала дадим определение эпитаксии: эпитаксия - это закономерное нарастание одного кристаллического материала на другой, то есть ориентированный рост одного кристалла на поверхности другого. В данной работе изучается процесс выращивания тонких монокристаллических полупроводниковых слоев на монокристаллических подложках. Процесс называют гомоэпитаксиальным, если материалы осаждаемого слоя и подложки идентичны, и гетероэпитаксиальным, если материалы осаждаемого слоя и подложки различаются. 
\section*{Слайд 3}

Развитие понимания физических процессов эпитаксиального роста и технологической базы привели к широкому использованию различных методов эпитаксиального роста в современной микро- и наноэлектроники.

Преимущества слоев, синтезированных по эпитаксиальной технологии приведены на слайде, а именно:
Широкая область изменения уровня и профиля легирования. Возможность изменения типа проводимости выращиваемых эпитаксиальных слоев. Физические свойства эпитаксиального слоя отличаются от свойств материала подложки в лучшую сторону, например, в них меньше концентрация кислорода и углерода, меньше число дефектов возможность проведения процесса эпитаксиального роста при температурах меньших, чем температура роста слитка монокристалла возможность нанесения эпитаксиального слоя как на большие площади, так и локально рост соединений со сложным, хорошо контролируемым составом
\section*{Слайд 4}
Существует три основных метода эпитаксиального роста: жидкофазная, газофазная и молекулярно-пучковая эпитаксия. При жидкофазной эпитаксии монокристаллические слои получают из контактирующих с подложкой пресыщенных жидких растворов. С повышением температуры избыточное количество растворенного материала осаждается из раствора на подложку, что связано с уменьшением растворимости материала. 
В методе газофазной эпитаксии вещество, необходимое для роста слоя, поступает к подложке в составе химического соединения, находящегося в газофой фазе. Вблизи подложки происходит разложение этого соединения (чаще всего пиролитическое), с выделением в качестве одного из продуктов разложения вещества, необходимого для роста эпитаксиальной пленки. Достоинствами этого метода являются высокая скорость роста слоев, высокая производительность и относительная дешевизна технологического оборудования. К недостаткам можно отнести токсичность большинства используемых материалов и зависимость скорости роста слоев от температуры подложки, которая выражается в значительном уменьшении скорости роста при уменьшении температуры подложки. 

В методе молекулярно-пучковой эпитаксии химические элементы, из которых состоит растущая пленка, поступают на подложку в виде атомарных или молекулярных пучков этих элементов. При этом для роста пленок высокого кристаллического качества необходимо проведения процесса роста в условиях сверхвысокого вакуума. Сущность метода состоит в соиспарении материалов, составляющих эпитаксиальный слой и легирующих примесей, их переносе на разогретую подложку и кристаллизация на ее поверхности. При этом отсутствуют промежуточные химические реакции в газовой фазе и значительные диффузные эффекты. Это позволяет быстро и управляемо менять состав растущего слоя. Достоинствами этого метода являются возможность роста пленок при пониженных температурах роста и хороший контроль над составом и толщиной осаждаемого слоя. Недостатки этого метода - дороговизна оборудования и низкую производительность. Именно метод молекулярно-пучковой эпитаксии рассматривается в данной работе.
\section*{Слайд 5}
В случае гетероэпитаксиального роста возможны различные механизмы роста в зависимости от параметров материалов подложки и осаждаемого слоя. В случае, когда параметры кристаллической решетки и подложки и осаждаемого материала совпадают и в растущей пленке не возникает механических напряжений, то механизм эпитаксиального роста зависит от поверхностных энергий подложки ($\gamma_1$ ), осаждаемого материала ($\gamma_2$) и энергии гетерограницы ($\gamma_{12}$).

\section*{Слайд 6}
Если выполняется соотношение (указать на слайд) выполняется механизм Франка-Ван-дер-Мерве. Двумерный рост пленки приводит к уменьшению суммарной энергии системы за счет уменьшения поверхностной энергии. При этом говорят, что осаждаемый материал смачивает подложку. Если выполняется обратное соотношение (*), то реализуется рост по механизму Вольмера-Вебера. В этом случае смачивание не происходит. 

\section*{Слайд 7}
Рост по механизму Странского-Крастанова рассмотри более подробно. Так как рост GeSi кристаллов происходит именно по этому механизму. Этот механизм реализуется при различных постоянных кристаллических решеток. На начальном этапе роста осаждаемый материал образует однородный слой, что приводит к уменьшению суммарной энергии системы. Однако при дальнейшем росте происходит возрастание энергетического слагаемого связанного с упругими напряжениями, вызванными рассогласованием кристаллических решеток пленки и подложки. С определенной толщины рост двумерной бездефектной пленки становится энергетически невыгоден, так как необходимо уменьшение этого слагаемого за счет релаксации упругих напряжений

\section*{Слайд 8}
Существует два механизма релаксации упругих напряжений: Классический и Когерентный.
При классическом механизме релаксации происходит образование дислокаций несоответствия. В месте образования дислокаций происходит уменьшение химического потенциала, что приводит к диффузии осаждаемого материала в эту область и к образованию трехмерного островка. Второй механизм релаксации [ебучий непонятный блог. Когерентный механизм релаксации.]

\section*{Слайд 9}
Известно, что на поверхности монокристаллической подложки имеются со свободными атомными связями. Для того, чтобы уменьшить число свободных связей и , как следствие, понизить суммарную энергию образуется кристаллическая перестройка, называемая реконструкцией поверхности. При реконструкции поверхности соседние атомы кремния
образуют так называемого Димана. Диманы в пределах одной атомной плоскости выстраиваются в цепочки. Из-за того, что кремний имеет кристаллографическую структуру алмаза, направление Диманов в каждом последующем слое меняется на $\frac{\pi}{2}$.

\section*{Слайд 10}
В результате рассогласования	параметром кристаллических решеток Ge и Si пленка при осаждении на поверхность испытывает упругие напряжения сжатия. Эти напряжения приводят к тому, что в цепочках Диманов образуются дивакансии (отсутствие в цепочке одного Димана). Вышеописанные эффекты приводят к образованию упорядоченной структуры. 

При увеличении Ge рост упругой энергии вызывает появление реконструкции поверхности, в которой к  упорядоченным рядам дивакансий добавляются перпендикулярные им линии дивакансий, представляющие собой отсутствие одного ряда Диманов.  

Дальнейшая релаксация упругих напряжений в растущей пленке происходит за счет появления ямок, образованных скоплением вакансий Диманов, в месте пересечения
упорядоченных линий дивакансий и рядов дивакансий. 
\section*{Слайд 11}

Дальнейшим энергетически выгодным 
шагом является образование островков на месте углублений, образованных пересечениями отсутствующих Диманов. Толщина пленки при которой начинается формирование островков называется критической толщиной двумерного роста. Эта величина зависит от состава пленки, температуры роста и скорости осаждения. При температуре больше $500^{\circ}$ критическая толщина пленки Ge 3-5 монослоев. 

\section*{Слайд 12} % (fold)
% \label{sec:слайд_12}


Первым этапов роста островком является образование предпирамид, имеющих плоскую форму с малым углом между боковыми гранями и плоскостью подложки. При этом основную роль играет дополнительная поверхностная энергия боковых граней островков.
По мере увеличения количества осаждаемого материала всё большую роль начинает играть слагаемое, связанное со степенью релаксации упругих напряжений. Степень релаксации упругих напряжений растет с ростом угла между боковыми гранями и подложкой. Поэтому по мере увеличения количества осаждаемого материала появляются островки с всё большим отношением высоты островка к его латеральному размеру. Несмотря на то, что такие островки имеют большую поверхностную энергию  по сравнению с плоскими.

Следовательно, предпирамиды при достижении некоторого критического объема трансформируются в усеченные пирамиды, которые затем приобретают пирамидальную форму. 


При высоких температурах проведения эксперимента вместо пирамид образуются HUT-структуры. Если продолжать увеличивать количество осажденного материала при соответственных (высоких) температурах неоднородности меняют свой тип на куполообразный. 
% section слайд_12 (end)

\section*{Описание установки} % (fold)
% \label{sec:описание_установки}
Как отмечалось выше, для получения высококачественных пленок в методе МПЭ необходимо проведение роста в условиях сверхвысокого вакуума. Для его достижения установка Riber Siva 21 состоит из трёх камер: загрузки, подготовки и роста с индивидуальными системами откачки и транспортом подложек между камерами. Камеры отделены друг от друга высоковакуумными затворами (шиберами). С помощью сложных системы насосов достигается глубокий вакуум $10^{-11}$ Торр. Нагрев подложки осуществляется за счёт радиационного излучения от печи. Максимальная температура нагрева составляет $1100^{\circ} $С. Контроль управления температуры подложки осуществляется с помощью термопары.  
% section описание_установки (end)
\end{document}
