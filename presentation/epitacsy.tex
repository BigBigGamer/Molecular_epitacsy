\documentclass[10pt,pdf,hyperref={unicode}, dvipsnames]{beamer}
\input{text/pre.tex}
\renewcommand{\phi}{\varphi}
\renewcommand{\epsilon}{\varepsilon}
\renewcommand{\div}{\operatorname{div}}
\title[КЕК]{КЕК}

\author{%
	Виноградов И.Д. %
	Понур К.А. %
	Шиков А.П. %
}

\institute{Радиофизический факультет ННГУ, 430 группа}

\date{Нижний Новгород, 2018}

\begin{document}  
\begin{frame}
\titlepage
\end{frame}


\section{Введение}
\subsection{Определение эпитаксии}
\begin{frame}[t]
	\frametitle{Определение эпитаксии}
	
\end{frame}


\subsection{Автоэпитаксия, гомоэпитаксия}
\begin{frame}[t]
	\frametitle{Автоэпитаксия, гомоэпитаксия}
\end{frame}




\section{Область применения}
\begin{frame}[t]
	\frametitle{Область применения}
\end{frame}





\section{Типы эпитаксии}
\subsection{Типы эпитаксии}
\begin{frame}[t]
	\frametitle{Типы эпитаксии}
\end{frame}

\subsection{Структурные дефекты поверхности}
\begin{frame}[t]
	\frametitle{Структурные дефекты поверхности}
\end{frame}



\section{Основные механизмы роста}


\section{Гетероэпитаксиальный рост}


\section{Релаксация упругих напряжений}


\section{Рост SiGe структур}
\subsection{Типы образуемх стуркутр}


\section{Эксперимент}


\end{document}
