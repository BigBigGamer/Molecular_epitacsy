\documentclass[10pt,pdf,hyperref={unicode}, dvipsnames]{beamer}
\input{text/pre.tex}
\renewcommand{\phi}{\varphi}
\renewcommand{\epsilon}{\varepsilon}
\renewcommand{\div}{\operatorname{div}}
\title[Метод молекулярно-пучковой эпитаксии]{Метод молекулярно-пучковой эпитаксии}

\author{%
	Виноградов И.Д. %
	Понур К.А. %
	Шиков А.П. %
}

\institute{Радиофизический факультет ННГУ, 430 группа}

\date{Нижний Новгород, 2018}

\begin{document}  
\begin{frame}
\titlepage
\end{frame}


\section{Введение}
\subsection{Определение эпитаксии}
\begin{frame}[t]
	\frametitle{Определение эпитаксии}
	\textbf{Эпитаксия} - это закономерное нарастание одного кристаллического материала на другой, т.е. ориентированный
	рост одного кристалла на поверхности другого.
	\vspace{20pt}

	\centering
	\begin{minipage}{0.49\linewidth}
		\centering
		\textbf{Авто(гомо)эпитаксия}

		Материалы осаждаемого слоя и подложки идентичны, или имеют одинаковую кристаллическую решетку
		\centering

		\includegraphics[width=0.6\linewidth]{imgs/SiCell.png}
		
		Решетка германия и кремния
	\end{minipage}
	\begin{minipage}{0.49\linewidth}
		\centering
		\textbf{Гетероэпитаксия}

		Материалы осаждаемого слоя и подложки различны
		\includegraphics[width=0.9\linewidth]{imgs/RutHem.jpg}
		Рутил на гематите
	\end{minipage}

\end{frame}


\subsection{Область применения}
\begin{frame}[t]
	\frametitle{Область применения}
	Эпитаксия является одним из базовых процессов технологии изготовления полупроводниковых приборов и интегральных
	схем.
	\vspace{20pt}

	Преимущества эпитаксиальной технологии
	\begin{enumerate}
		\item Широкая область изменения уровня и профля легирования 
		\item Возможность изменения типа проводимости выращиваемых эпитаксиальных слоев
		\item Возможность проведения роста при температурах меньших, чем температура роста монокристалла
		\item Возможность нанесения слоя как на больщие площади, так и локально
		\item Рост соединений со сложным, контролируемым составом
	\end{enumerate}

\end{frame}



\section{Методы эпитаксиального роста}
\subsection{Методы эпитаксиального роста}
\begin{frame}[t]
	\frametitle{Методы эпитаксиального роста}
	Существует три основных метода эпитаксиального роста: \textit{Жидкофазная}, \textit{газофазная} и
	\textit{молекулярно-пучковая} эпитаксия.
	\vspace{10pt}

	Жидкофазная: Монокристаллические слои получают из контактирующей с подложкой перенасыщенных жидких растворов.

	\textbf{Недостатки}: Сложности контроля параметров получаемых пленок, низкое качество. 
	\vspace{10pt}

	Газофазная: Вещество, необходимое для роста поступает к подложке в составе химического соединения, с выделением при
	разложении(обычно термическом) вещества, необходимого для роста эпитаксиальной пленки.

	\textbf{Достоинства}: Высокая скорость роста, высокая производительность.

	\textbf{Недостатки}: Токсичность, зависимость скорости роста от температуры подложки.
	\vspace{10pt}

	Молекулярно-лучевая: Хим. элементы, необходимые для роста поступают на подложку в виде молекулярных пучков этих
	элементов.
	
	\textbf{Достоинства}: Возможность роста при пониженных температурах, лучший по сравнению с ГФЭ контроль над составом и
	толщиной слоев.

	\textbf{Недостатки}: Низкая производительность, дороговизна(необходим сверхвакуум).
\end{frame}

\section{Основы МПЭ}
\subsection{Структурные дефекты поверхности}
\begin{frame}[t]
	\frametitle{Структурные дефекты поверхности}
\end{frame}



\section{Основные механизмы роста}


\section{Гетероэпитаксиальный рост}


\section{Релаксация упругих напряжений}


\section{Рост SiGe структур}
\subsection{Типы образуемх стуркутр}


\section{Эксперимент}


\end{document}
